\documentclass[colorback, accentcolor=tud9d, paper=a4]{tudreport}
\usepackage[ngerman]{babel}
\usepackage[utf8]{inputenc}
\usepackage{graphicx,tabularx,array}


\begin{document}
\newcolumntype{P}[1]{>{\centering\arraybackslash}p{ #1}}
\setlength{\unitlength}{\textwidth}
\begin{minipage}{0.6\textwidth}
\large Dokumentation der täglichen Arbeitszeit nach\\
§14 Mindeslohngesetz für Minijobber
\end{minipage}
\begin{minipage}{0.4\textwidth}
	\includegraphics[scale=0.27]{logo.png}
	
\end{minipage}
\begin{center}
		\parbox{40em}{
			\begin{flushleft}
			\color{red}WICHTIG:\\
			Die  Aufzeichnungen  sind  mindestens  wöchentlich  zu  führen,  denn  der 
			Arbeitgeber  "`ist  verpflichtet,  Beginn,  Ende  und  Dauer  der  täglichen 
			Arbeitszeit dieser Arbeitnehmerinnen und Arbeitnehm	er spätestens bis zum 
			Ablauf des siebten auf den Tag der Arbeitsleistung 
			folgenden Kalendertages 
			aufzuzeichnen und diese Aufzeichnungen mindestens zwei Jahre beginnend 
			ab dem für die Aufzeichnung maßgeblichen Zeitpunkt 
			aufzubewahren"'. 
		\end{flushleft}	
	}
\end{center}
\vspace{1em}
\renewcommand{\arraystretch}{1.8}
\begin{tabular}{l p{33em}}
	\textbf{Einrichtung (Institut)}: & {{institute}}\\\cline{2-2}\noalign{\smallskip}
	\textbf{Name, Vorname}: & {{name}}, {{firstname}}  \\\cline{2-2}\noalign{\smallskip}
	\textbf{Geb. Datum}: &  {{birthday}} \\\cline{2-2}\noalign{\smallskip}
\end{tabular}
\\[3em]
\begin{tabular}{l |p{14em} |p{14em}|}
	\cline{2-3}
	\textbf{Aufzeichnung für den Zeitraum} & \textbf{vom} {{ period.start }}& \textbf{bis} {{ period.end }} \\\cline{2-3}
	
\end{tabular}
\\[1em]
\begin{tabularx}{\textwidth}{|c|c|c|P{4em}|X|}
	\hline
	\textbf{Datum} & \textbf{Beginn} & \textbf{Ende}& \textbf{gesamt Stunden} & \textbf{Unterschrift Mitarbeiter*in}\\\hline
	
		{{ day.day }} & {{ day.start }} & {{ day.end }} & {{ day.hours }} & \\\hline
	
		
	& & & &  \\\hline
	
\end{tabularx}
\\[4em]
\begin{minipage}{0.35\textwidth}
	Darmstadt, den \dotfill
\end{minipage}
\\[5em]
\begin{minipage}{0.35\textwidth}
	\begin{center}
		\dotfill\\
	\textbf{Unterschrift der/des Vorgesetzten}
	\end{center}
\end{minipage}\hspace{3cm}
\begin{minipage}{0.35\textwidth}
	\vspace{1em}
	\textbf{Stempel der Einrichtung/ des Instituts}
\end{minipage}


\end{document}
